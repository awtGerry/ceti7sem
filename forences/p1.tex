\documentclass{article}

\usepackage[utf8]{inputenc}
\usepackage[T1]{fontenc}
\usepackage[english]{babel}
\usepackage[document]{ragged2e}
\usepackage{amsfonts}
\usepackage{natbib}
\usepackage[dvipsnames]{xcolor}
\usepackage{graphicx}
\usepackage{fancyhdr}

\usepackage[fleqn]{amsmath}
\usepackage{amssymb}

\pagestyle{fancy}
\lhead{\includegraphics[width=0.5cm]{~/Documents/ceti/Logo-CETI-Black.jpg}}
\rhead{Practica 01}
\cfoot{\thepage}
\renewcommand{\headrulewidth}{0.4pt}
\renewcommand{\footrulewidth}{0.4pt}

\title{Practica 01}
\author{Victor Gerardo Rodríguez Barragán}
\date{21 de mayo de 2024}

\begin{document}
\maketitle

El archivo ha sido modificado lo podemos saber
por el sha256 que no es el mismo que el original.
\begin{center}
    \includegraphics[width=0.5\textwidth]{~/Pictures/screenshots/area/screenshot-2024-05-21_12-06-57.png}
\end{center}

\begin{itemize}
    \item CRC-32b es un algoritmo de suma de verificación rápido y ligero que se usa para detectar errores accidentales en datos. No es seguro contra manipulaciones intencionales.
    \item SHA-256 es una función hash criptográfica que produce un hash de 256 bits. Es mucho más robusto y seguro contra manipulaciones intencionales.
\end{itemize}

\end{document}
