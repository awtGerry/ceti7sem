\documentclass{article}

\usepackage[utf8]{inputenc}
\usepackage[T1]{fontenc}
\usepackage[english]{babel}
\usepackage[document]{ragged2e}
\usepackage{amsfonts}
\usepackage{natbib}
\usepackage[dvipsnames]{xcolor}
\usepackage{graphicx}
\usepackage{fancyhdr}

\usepackage[fleqn]{amsmath}
\usepackage{amssymb}

\pagestyle{fancy}
\lhead{\includegraphics[width=0.5cm]{~/Documents/ceti/Logo-CETI-Black.jpg}}
\rhead{Internet de las cosas IoT}
\cfoot{\thepage}
\renewcommand{\headrulewidth}{0.4pt}
\renewcommand{\footrulewidth}{0.4pt}

\title{Process Optimization through Virtual Visibility}
\author{Victor Gerardo Rodríguez Barragán}
\date{June 04, 2024}

\begin{document}
\maketitle
\includegraphics[width=\textwidth]{~/Documents/ceti/ceti_logo.png}
\newpage
\justify
\section{Introduction}
First he introduced the concept of the company, he explain that it was not a ``tech focused''
company, but they were trying to implement technology to improve their processes. He mentioned
that they were trying to implement a new system to improve the visibility of their processes.
\vspace{0.5cm}\\
The company itself was like a vegetables processing company, they had a lot of processes and he
tried to explain something about corn, which I didn't completely understand the purpose of it, but
he did, he said that corn is a very important product in life.

\section{Content}
He mentioned that they were trying to implement new technology to improve the overall operation of
the different processes in the company, one of them was a better data collection and management, which
includes data analysis and machine learning, this two being the key to the vision that they had.
\vspace{0.5cm}\\
Something that caught my attention was the tech (like the software or computer language) that they were
trying to implement, he mentioned that they were using Python, which is a very popular language for
what they were trying to do, and that they were using a lot of libraries like Pandas, Numpy, and Scikit-learn, etc, which
in like a general opinion it's nice, it is what you would expect from a company that is trying to accomplish what
they do. He also mentioned matlab and R, I would love to know more about the use of this two languages in the company.
Since Python is already ``enough'' for the goal.

\section{Conclusion}
In general I liked the presentation, I think that the company is doing a good job trying to implement new technology
especially right now with AI as strong as it is, I think that they are in the right path, for that point it cougth my
attention but I think it was a little bit long for what he was trying to explain, I really think that a shorter presentation
could have the same impact but with a stronger message and with the 100\% of the attention of the audience (but that's just my opinion
I'm a very easy distracted person).

\section{Question}
Well for me the best way to do it is to use a scanner, you can scan the data and then
use a OCR software to convert it to a digital format, this may have troubles since it's not
100\% accurate, but it's the best way to do it, because even if I can type all the data by hand
it's not the best way to do it, it's very time consuming and it's not very efficient, and we
as a programmers are always looking for the most efficient way to do things.

The main goal not only for us but for a company is to be efficient, to have the best results
in the less time possible, and that's why we are here for, to automate processes, and
try to make the best results in general.

\end{document}
