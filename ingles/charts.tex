% Title: Analysis of Average Monthly Temperatures for Three African Cities
\documentclass{article}

\usepackage[utf8]{inputenc}
\usepackage[T1]{fontenc}
\usepackage[english]{babel}
\usepackage[document]{ragged2e}
\usepackage{amsfonts}
\usepackage{natbib}
\usepackage[dvipsnames]{xcolor}
\usepackage{graphicx}
\usepackage{fancyhdr}

\usepackage[fleqn]{amsmath}
\usepackage{amssymb}

\pagestyle{fancy}
\lhead{\includegraphics[width=0.5cm]{~/Documents/ceti/Logo-CETI-Black.jpg}}
\rhead{Internet de las cosas IoT}
\cfoot{\thepage}
\renewcommand{\headrulewidth}{0.4pt}
\renewcommand{\footrulewidth}{0.4pt}

\title{AWS}
\author{Victor Gerardo Rodríguez Barragán}
\date{11 de abril de 2024}

\begin{document}
\maketitle
\includegraphics[width=\textwidth]{~/Documents/ceti/ceti_logo.png}
\newpage
\justify
Analysis of Average Monthly Temperatures for Three African Cities

The graph illustrates the average monthly temperatures for three African cities: Mombasa (Kenya), Cairo (Egypt), and Cape Town (South Africa). The temperatures are measured in degrees Fahrenheit, with the Y-axis ranging from 40 to 85, and the X-axis representing the months of the year.
Key Features

\begin{itemize}
    \item Mombasa, Kenya (Red Line):
        Starts at 80°F in March, peaks at 82.5°F in April, and gradually decreases to 80°F by December.
        Shows a slight fluctuation, with the highest temperature in April and the lowest in July.
    \item Cairo, Egypt (Blue Line):
        Begins at 55°F in February, rises significantly to 81°F in July and August, and then decreases to 58°F by December.
        Experiences a sharp increase in temperatures from February to July, followed by a gradual decline.
    \item Cape Town, South Africa (Gray Line):
        Starts at 70°F in February, decreases steadily to 55°F in July, and then increases slightly to 67.5°F by December.
        Shows a consistent decline in temperatures from February to July, followed by a gradual increase towards the end of the year.
\end{itemize}

Mombasa experiences relatively stable temperatures throughout the year, with a slight peak in April.
Cairo has a significant temperature increase from February to July, with the highest temperatures recorded in July and August.
Cape Town experiences a steady decline in temperatures from February to July, followed by a gradual increase towards the end of the year.
\vspace{0.5cm}\\
Mombasa has a relatively stable climate throughout the year, with temperatures ranging from 75°F to 82.5°F.
Cairo has a hot climate, with temperatures ranging from 55°F to 81°F, peaking in July and August.
Cape Town has a mild climate compared to the other cities, with temperatures ranging from 55°F to 70°F, reaching its lowest point in July.
\vspace{0.5cm}\\
The other chart compares the percentages of different home heating sources in 1950 and 1997.
The Y-axis represents the percentage range from 0.0\% to 60.0\%, while the X-axis lists the
heating sources: Coal, Natural Gas, Propane, Fuel Oil, Kerosene, Electricity, Wood, No heat source, and Other.
Key Features

\begin{itemize}
    \item
    Coal:
        Accounted for 33\% of home heating sources in 1950, but was no longer used by 1997.
    \item
    Natural Gas:
        Increased from 25\% in 1950 to 51\% in 1997, showing a significant upward trend.
    \item
    Propane:
        Increased from 2.5\% in 1950 to 6\% in 1997, indicating a moderate increase.
    \item
    Fuel Oil:
        Decreased from 22\% in 1950 to 10\% in 1997, showing a notable decline.
    \item
    Kerosene:
        Decreased from 3.5\% in 1950 to 0.5\% in 1997, indicating a substantial decline.
    \item
    Electricity:
        Increased from 0.5\% in 1950 to 29.5\% in 1997, showing a significant rise.
    \item
    Wood:
        Decreased from 9.5\% in 1950 to 2.5\% in 1997, indicating a significant decline.
    \item
    No Heat Source:
        Decreased from 3.5\% in 1950 to 0.5\% in 1997, showing a notable decline.
    \item
    Other:
        Accounted for 2\% of home heating sources in 1950 but was not used in 1997.
\end{itemize}
The use of Coal, Kerosene, and Wood as home heating sources declined significantly from 1950 to 1997.
Natural Gas and Electricity saw significant increases as heating sources during the same period.
Propane and Fuel Oil also experienced moderate changes in usage.
\vspace{0.5cm}\\
The shift in home heating sources over the years reflects changes in technology, availability, and environmental awareness.
Natural Gas and Electricity became more popular due to their convenience and cleanliness compared to Coal, Wood, and Oil-based fuels.
The decline of traditional sources like Coal and Wood may indicate a move towards more efficient and environmentally friendly heating options.


\end{document}
