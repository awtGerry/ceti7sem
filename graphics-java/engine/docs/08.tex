\documentclass{article}

\usepackage[utf8]{inputenc}
\usepackage[T1]{fontenc}
\usepackage[english]{babel}
\usepackage[document]{ragged2e}
\usepackage{amsfonts}
\usepackage{natbib}
\usepackage[dvipsnames]{xcolor}
\usepackage{graphicx}

\usepackage[fleqn]{amsmath}
\usepackage{amssymb}

\title{Circunferencia de coordenadas polares}
\author{Victor Gerardo Rodríguez Barragán}
\date{}

\begin{document}
\maketitle
\justify

\begin{itemize}
\item
Al calcular cada punto del círculo utilizando coordenadas polares, el algoritmo realiza
muchos cálculos trigonométricos (seno y coseno) en cada iteración del bucle.
Esto puede ser muy ineficiente en comparación con otros métodos.
\item
Uso excesivo de memoria y mayores tiempos de ejecución.
\item
Debido a la naturaleza de los cálculos es posible que este algoritmo tenga errores de
redondeo que afecten la precisión de los puntos dibujados, especialmente en los bordes del círculo.
\end{itemize}
\end{document}
