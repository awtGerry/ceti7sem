\documentclass{article}

\usepackage[utf8]{inputenc}
\usepackage[T1]{fontenc}
\usepackage[english]{babel}
\usepackage[document]{ragged2e}
\usepackage{amsfonts}
\usepackage{natbib}
\usepackage[dvipsnames]{xcolor}
\usepackage{graphicx}

\usepackage[fleqn]{amsmath}
\usepackage{amssymb}

\title{Circunferencia}
\author{Victor Gerardo Rodríguez Barragán}
\date{}

\begin{document}
\maketitle
\justify

\begin{itemize}
\item La ecuación puede llevar a una cantidad significativa de cálculos repetitivos y poco eficientes,
    especialmente para círculos grandes.
\item La precisión de la circunferencia es limitada.
\item El algoritmo dibuja tanto topY como bottomY para cada valor de x, lo que puede resultar en la
    duplicación de puntos dibujados en el círculo, lo que consume más recursos y puede afectar la
    apariencia visual del círculo (por ejemplo, el circulo sale con saltos de píxeles).
\end{itemize}
\end{document}
