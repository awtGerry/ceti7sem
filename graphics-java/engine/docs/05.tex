\documentclass{article}

\usepackage[utf8]{inputenc}
\usepackage[T1]{fontenc}
\usepackage[english]{babel}
\usepackage[document]{ragged2e}
\usepackage{amsfonts}
\usepackage{natbib}
\usepackage[dvipsnames]{xcolor}
\usepackage{graphicx}

\usepackage[fleqn]{amsmath}
\usepackage{amssymb}

\title{Diferencias entre el algoritmo de Bresenham y el algoritmo de punto medio.}
\author{Victor Gerardo Rodríguez Barragán}
\date{}

\begin{document}
\maketitle
\justify

En el algoritmo de Bresenham, se utiliza el valor $p = 2 * dy - dx$ para determinar cuándo cambiar
de píxel en el eje $y$. Este valor se actualiza según una decisión basada en $p$ en cada paso. Ademas
se utilizan incrementos constantes (\textbf{xInc} y \textbf{yInc}) que dependen de la dirección de la línea. La actualización
de $p$ también depende de la dirección de la línea y se calcula de manera diferente según si la pendiente es mayor a 1 o menor a 1.
\vspace{0.25cm}\\
En el algoritmo de punto medio, se utiliza el valor $d = dy - (dx / 2)$ para determinar cuándo cambiar
de píxel en el eje $y$. Este valor también se actualiza en cada paso, pero la forma en que se actualiza es diferente a la de Bresenham.
Los incrementos $(x y)$ son constantes y no dependen de la dirección de la línea. La actualización de $d$ se realiza de manera diferente
dependiendo de si el píxel seleccionado está arriba o abajo de la línea.
\end{document}
