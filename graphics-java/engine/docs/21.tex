\documentclass{article}

\usepackage[utf8]{inputenc}
\usepackage[T1]{fontenc}
\usepackage[english]{babel}
\usepackage[document]{ragged2e}
\usepackage{amsfonts}
\usepackage{natbib}
\usepackage[dvipsnames]{xcolor}
\usepackage{graphicx}

\usepackage[fleqn]{amsmath}
\usepackage{amssymb}

\title{Circunferencia de coordenadas polares}
\author{Victor Gerardo Rodríguez Barragán}
\date{}

\begin{document}
\maketitle
\justify

El algoritmo de Sutherland-Hodgman se utiliza para recortar polígonos en los bordes de la región visible.
Dada una lista de entrada de vértices que definen un polígono (por ejemplo, una fila de la matriz de caras),
el algoritmo de Sutherland-Hodgman recorre cada vértice del polígono y si el vértice actual está dentro de la región visible,
verificamos si el vértice anterior también está dentro.
Si es la línea que une los dos puntos está completamente dentro y agregamos ambos puntos finales a una lista de salida
que define el polígono recortado y pasamos al siguiente vértice en la lista de entrada.
Si el vértice anterior no está dentro de la región visible, recortamos la línea usando el algoritmo de Cyrus-Beck y agregamos
el punto recortado y el punto final actual a la lista de salida. Si el punto actual no está dentro de la región visible pero el anterior sí,
recortamos la línea hasta el borde y agregamos el punto recortado a la lista de salida.
Una vez que se han considerado todos los vértices de la lista de entrada, la lista de salida contiene la información del polígono recortado.

\end{document}
