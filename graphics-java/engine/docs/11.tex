\documentclass{article}

\usepackage[utf8]{inputenc}
\usepackage[T1]{fontenc}
\usepackage[english]{babel}
\usepackage[document]{ragged2e}
\usepackage{amsfonts}
\usepackage{natbib}
\usepackage[dvipsnames]{xcolor}
\usepackage{graphicx}

\usepackage[fleqn]{amsmath}
\usepackage{amssymb}

\title{Circunferencia de coordenadas polares}
\author{Victor Gerardo Rodríguez Barragán}
\date{}

\begin{document}
\maketitle
\justify

\begin{itemize}
\item
Este algoritmo también utiliza el método de Bresenham para dibujar el círculo, similar al algoritmo de punto medio.
\item
Utiliza una condición de decisión para determinar cómo actualizar el valor de p en cada iteración del bucle, lo que permite dibujar el círculo de manera eficiente y sin puntos superpuestos.
\item
Existen diferencias en la forma en que se calcula p, lo que puede afectar ligeramente la precisión o la eficiencia en comparación con el algoritmo de punto medio.

En resumen, los tres algoritmos tienen como objetivo dibujar un círculo, pero difieren en la forma en que calculan
los puntos del círculo y cómo evitan la superposición de puntos. El algoritmo de Bresenham y el algoritmo de punto medio
suelen ser más eficientes y precisos que el algoritmo de coordenadas polares, o el base.

\end{itemize}
\end{document}
