\documentclass{article}

\usepackage[utf8]{inputenc}
\usepackage[T1]{fontenc}
\usepackage[english]{babel}
\usepackage[document]{ragged2e}
\usepackage{amsfonts}
\usepackage{natbib}
\usepackage[dvipsnames]{xcolor}
\usepackage{graphicx}

\usepackage[fleqn]{amsmath}
\usepackage{amssymb}

\title{Modelo matemático mejorado para dibujar una línea recta}
\author{Victor Gerardo Rodríguez Barragán}
\date{}

\begin{document}
\maketitle
\justify

El algoritmo principal comparte la misma estructura base
para dibujar una linea entre los puntos $x_1,y_1$ y $x_2,y_2$.
\vspace{0.25cm}\\
En el codigo manejamos ahora casos especificos, como el caso
el valor de $dx = 0$ o $dy = 0$ para evitar divisiones estas
divisiones y mejorar la eficiencia. Por ejemplo, si dx es cero,
significa que la línea es vertical, por lo que se dibuja de
forma vertical incrementando solo el eje y.
\vspace{0.25cm}\\
Pero la mejora más importante es la condicional si $dx$ es mayor que $dy$ donde
se compara la magnitud de $dx$ y $dy$ para determinar qué eje (horizontal o vertical)
incrementar durante el dibujo de la línea. Esto ayuda a mejorar la eficiencia y la precisión del dibujo
en casos donde la línea es más vertical que horizontal, o viceversa y no solo una línea diagonal perfecta.
\end{document}
